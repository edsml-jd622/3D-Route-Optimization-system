% Document class `report-template` accepts either project-plan or final-report option in [].
\documentclass[final-report]{report-template}

% Packages I use in my report.
\usepackage{graphicx}
\usepackage{amsmath}
\usepackage{blindtext}

% Directory where I saved my figures.
\graphicspath{{./figures/}}

% Metadata used for the title page - please modify.
\university{Imperial College London}
\department{Department of Earth Science and Engineering}
\course{MSc in Environmental Data Science and Machine Learning}
\title{Building a route optimisation system that takes elevation into consideration}
\author{Jinsong Dong}
\email{jinsong.dong22@imperial.ac.uk}
\githubusername{edsml-jd622}
\supervisors{Sesinam Dagadu\\
             Dr. Yves Plancherel}
\repository{https://github.com/ese-msc-2022/irp-jd622}

\begin{document}

\maketitlepage  % generate title page

% Abstract
\section* {Abstract}
Roughly 200-word.

% Introduction section
\section {Introduction}
\subsection {Discription of the problem}
Summary of the problem.\\
Brief literature review.\\
Objectives and Hypotheses.\\
Tasks.\\
\subsection {Literature Review}
State-of-the-art solutions to the problem (including commercial and academic approaches).\\
How my project go beyond the state-of-the-art solutions, and original work I've done.\\

\section {Methods}
Technical back-end of solutions, describe if it is standalone code or an extension of a pre-existing code and ecosystem I made.\\
List development, operation tools, development methodologies, why.\\
Architectural design diagram of your solution if relevant.\\
Design rationale, implementation strategy, data structure, routine, verification and validation.\\
Algorithm, pseudo-code.\\
Creativity\\
Implementation platform, programming language, libraries.\\

\subsection {Building 3D road network}
Collecting 2D road information from OpenstreetMap, collecting elevation data from **** website. 
Integrate these two information tegother to do interpolation so that every node has assigned its elevation value.
!The figure to show how to do interpolation.
The integrate 3D road data is in json format.

Building network based on the point, tag data of the 3D road json file with NetworkX to build a graph network.
This network can be used to calculate several information between any pairs of points in the graph.
Information including 2D distance, 3D distance, traveling time. And this information can be added according to the purpose of the user.

\subsection {Pointer Network for TSP problem of 3D road network}




\section {Results}
Quantify the results.\\

\section {Discussion and Conclusions}
The most difficult tasks to resolve.\\
Strengths and limitations.\\
Next steps.\\




% References
\bibliographystyle{unsrt}
\bibliography{references}  % BibTeX references are saved in references.bib

\end{document}          
